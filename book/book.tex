\documentclass[10.5pt,a5paper,twoside]{memoir}
\usepackage[german]{babel}
\usepackage[utf8]{inputenc} 

\usepackage{csquotes}

\renewcommand{\thesection}{\arabic{section}}
\setcounter{secnumdepth}{3} % Subsection mit Zähler (1.1) versehen

%% pt sans
\usepackage[T1]{fontenc}
\usepackage{PTSerif}

\usepackage{textcomp}

%%% Other Settings %%%
\sloppy
\raggedbottom

\usepackage{rotating}
%%% disable figure numbering

%%% import
\usepackage{import}

%%% captions
\usepackage{caption}

%%% graphics
\usepackage{graphicx}
\setkeys{Gin}{width=.8\textwidth} %default pics size
\usepackage{longtable}

%%%%url
\usepackage[hyphens]{url}
\urlstyle{same}
\usepackage[colorlinks, linkcolor=black,citecolor=black, urlcolor=blue,
breaklinks= true]{hyperref}
%%% remove chapter

\renewcommand*{\chaptitlefont}{\LARGE}

%%%%%%%bod test
\linespread{1.15}

%%% microtype
\usepackage{microtype}


% Hurenkinder und Schusterjungen verhindern
\clubpenalty10000
\widowpenalty10000
\displaywidowpenalty=10000

%%
% Kolumnentitel
%%
\renewcommand{\chaptermark}[1]{ \markboth{#1}{}  } % Stil der Kopfzeile zurücksetzen
\renewcommand{\sectionmark}[1]{ \markright{#1}{} } % Stil der Kopfzeile 

\renewcommand*{\chaptitlefont}{\normalfont\LARGE\flushright}

% Page style
\makepagestyle{mystyle}
\makeevenhead{mystyle}{\small\thepage}{\small \#l10j}{}
\makeoddhead{mystyle}{}{\small\slshape\leftmark}{\small\thepage}
\pagestyle{mystyle}

%%
% Inhaltsverzeichnis
%%
\renewcommand*{\cftchapterdotsep}{\cftdotsep}
\settocdepth{chapter}
\renewcommand{\cftchapterfont}{\normalfont}
\renewcommand{\cftchapterpagefont}{\normalfont}

\makeatletter
\DeclareRobustCommand\authortoctext[1]{%
{\addvspace{20pt}\nopagebreak\leftskip0em\relax
\rightskip \@tocrmarg\relax
\noindent\itshape#1\par\addvspace{-7pt}}}
\makeatother
\newcommand\authortoc[1]{%
  \gdef\chapterauthor{#1}%
  \addtocontents{toc}{\authortoctext{#1}}}

\makeatletter
\renewcommand{\cfttocbeforelisthook}{\pagestyle{empty}\let\ps@plain\ps@empty}
\renewcommand{\cfttocafterlisthook}{\cleardoublepage\pagestyle{headings}}
\makeatother

\newlength\drop
\makeatletter
\newcommand*\titleM{\begingroup% Misericords, T&H p 153
\setlength\drop{0.08\textheight}
\centering
\vspace*{\drop}
{\Huge\bfseries Die Bibliothek als Idee}\\[\baselineskip]
{\scshape Beiträge des Symposiums \\ -- 10 Jahre LIBREAS. Library Ideas -- \\ am ICI Kulturlabor Berlin}\\[\baselineskip]
\vfill
{\large\scshape {\small herausgegeben vom}\\{\large LIBREAS. Verein e.V.}}\par
\vfill
{\scshape Berlin, Chur, Dresden, Exeter, München}\par
\vfill
{\scshape 2016}\par
%\vspace*{2\drop}
\endgroup}
%\makeatother

%%
% Dokumentenbeginn
%%
\begin{document}

\frontmatter


\pagestyle{empty}
\begin{flushright}
{\LARGE DIE BIBLIOTHEK ALS IDEE} 
\end{flushright}

\cleartorecto

\pagestyle{empty}
\titleM
\cleartoverso


\begingroup
\footnotesize
\parindent 0pt
\parskip \baselineskip
\vspace*{\fill}

LIBREAS. Verein e.V.(Hrsg.), Die Bibliothek als Idee : Beiträge des Symposiums \enquote{10
Jahre LIBREAS. Library Ideas} am ICI Kulturlabor Berlin. Berlin, 2016.

\textcopyright{}
2015 / 2016 Die Autorinnen und Autoren. 

Dieses Werk ist lizenziert unter einer Creative Commons Namensnennung 3.0 Lizenz.

Satz und Layout: Redaktion LIBREAS unter Verwendung von pandoc und \LaTeX. 

Druck: epubli.

Webausgabe: \url{http://libreas.eu/ausgabe28/}

Quellcoderepository: \url{https://github.com/libreas/ausgabe28/}

LIBREAS. Library Ideas sind: Ben Kaden, Maxi Kindling, Manuela Schulz (
Herausgeberinnen und Herausgeber), Linda Freyberg, Najko Jahn, Karsten
Schuldt, Matti Stöhr, Doreen Thiede (Redaktion).

    \endgroup
\newpage

\begin{flushright} Wir bedanken uns herzlich bei der |a|S|tec| angewandte
Systemtechnik GmbH für die finanzielle Unterstützung des Drucks.

\vspace{10mm}

Unser Dank gilt überdies der Firma Citavi und an das ICI Kulturlabor Berlin,
die das Symposium unterstützt haben.

\vspace{10mm}

Ein ebenso herzlicher Dank gebührt unseren Referentinnen und Referenten und
zahlreichen Gästen, die nicht nur aus dem Bibliothekswesen stammen und das
Symposium zu einem lebhaften Austauschort machten. Ein besonderer Dank gilt
Corinna Haas, die uns mit dem ICI Kulturlabor eine großartige Location
vermittelt und für den Erfolg des Symposiums massgeblich verantwortlich war.
Allen Unterstützern, die uns vor, während und nach dem Symposium tatkräftig
halfen, danken wir ebenso herzlich: ohne Euch hätten wir es nicht geschafft.

\vspace{10mm}

Wir widmen den Tagungsband den Mitglieder\_innen des LIBREAS. Verein e.V.
\end{flushright}

\cleartorecto

\tableofcontents*

\mainmatter
\pagestyle{mystyle}

%% Editorial

\authortoc{Redaktion LIBREAS}
\chapter*[Editorial]{Editorial}
\addcontentsline{toc}{chapter}{Editorial}
\refstepcounter{chapter}
\begin{flushright}
{\large Redaktion LIBREAS}
\end{flushright}

\vspace{5mm}

\import{../00editorial/}{master.tex}

%% Karin Aleksander
\authortoc{Karin Aleksander}
\chapter*[Ist eine transdisziplinäre Bibliothek möglich?]{Ist eine transdisziplinäre Bibliothek möglich? Oder: Wie die Geschlechterforschung Idee und Ideal der Bibliothek herausfordert}
\addcontentsline{toc}{chapter}{Ist eine transdisziplinäre Bibliothek möglich? Oder: Wie die Geschlechterforschung Idee und Ideal der Bibliothek herausfordert}
\refstepcounter{chapter}
\begin{flushright}
{\large Karin Aleksander}
\end{flushright}

\vspace{5mm}

\import{../01aleksander/}{master.tex}

%% Ute Engelkenmeier
\authortoc{Ute Engelkenmeier}
\chapter*[Bibliothek und Bibliothekare in Comedy und Komödie]{Das Bild der Bibliothek und Bibliothekare in den Fernsehgenres Comedy und Komödie}
\addcontentsline{toc}{chapter}{Das Bild der Bibliothek und Bibliothekare in den Fernsehgenres Comedy und Komödie}
\refstepcounter{chapter}
\begin{flushright}
{\large Ute Engelkenmeier}
\end{flushright}

\vspace{5mm}

\section*{ABSTRACT}

{\small Mittels der empirischen Methode der standardisierten, quantitativen
Inhaltsanalyse werden 51 Fernsehsendungen aus den Subgenres Comedy und
Komödie untersucht, in welchen Bibliotheken als Schauplatz und
Bibliothekare als Figur vorkommen. Der betrachtete Zeitraum der
Ausstrahlungen ist von Januar 2010 bis Juli 2015. Ziel der Untersuchung
ist die Klärung der Frage, ob Bibliotheken und Bibliothekare stereotyp
dargestellt werden und ob sie innerhalb der gewählten Subgenres Objekt
des Witzes sind. Untersucht werden unter anderem die Dimensionen
Erscheinungsbild und Nutzungsmotive. Die Untersuchung kommt zu dem
Schluss, dass das Erscheinungsbild von Bibliotheken überwiegend auf das
konstitutive Element der Bücherregale reduziert ist. In der Nutzung
überwiegen Motive wie Ausleihen, Lesen, aber auch Nutzung der Bibliothek
als sozialer Treffpunkt (Kennenlernen, Date) oder die Suche nach einem
ruhigen Rückzugsort. Das Bild der Bibliothekare ist im äußeren
Erscheinungsbild zwar konservativer als andere Figuren, jedoch nicht
negativ, auch sind die Rollen nicht Objekt des Witzes. Eine Tendenz zu
einer weniger konservativen Darstellung in jüngeren Produktionen ist
festzustellen.}

\newpage 

\import{../02engelkenmeier/}{master.tex}

%%% Hartmann
\authortoc{\\Frank Hartmann}
\chapter*[Dokumentation als Gegenidee zur Bibliothek]{Dokumentation als Gegenidee zur Bibliothek}
\addcontentsline{toc}{chapter}{Dokumentation als Gegenidee zur Bibliothek}
\refstepcounter{chapter}
\begin{flushright}
{\large Frank Hartmann}
\end{flushright}

\vspace{5mm}

\section*{ABSTRACT}

{\small
	Schon zur Wende ins 20. Jahrhundert zweifelte der belgische Privatgelehrte Paul Otlet an der Zukunft des Buches und der Bibliothek. Statt dessen begann er damit, eine Dokumentation und Neuorganisation des Weltwissens anzulegen, und mittels eines Karteikartensystems (Répertoire Bibliographique Universel) zu vernetzen. Dieses Projekt eines flexiblen, abfrageorientierten Wissensbestandes in einem \enquote{Hypermedium} (Otlet) besetzte jene technologische Leerstelle, die inzwischen eine die bibliothekarische Epoche aufsprengende neue Wissenskultur der digitalen Medialität produziert hat.}

\vspace{5mm}

\import{../04hartmann/}{master.tex}

%%% wagner
\authortoc{Kirsten Wagner}
\chapter*[Die architektonische Idee der modernen Bibliothek]{Die architektonische Idee der modernen Bibliothek}
\addcontentsline{toc}{chapter}{Die architektonische Idee der modernen Bibliothek}
\refstepcounter{chapter}
\begin{flushright}
{\large Kirsten Wagner}
\end{flushright}

\vspace{5mm}

\section*{ABSTRACT}

{\small
	Die moderne Bibliothek geht auf einen Übergang der Saalbibliothek zur 
Magazinbibliothek zurück, der sich im 19. Jahrhundert vollzieht. Bevor
erste Magazinbibliotheken ab den 1850er Jahren gebaut werden, existiert
die moderne Bibliothek in der ersten Hälfte des 19. Jahrhunderts vor
allem als Idee. Es entsteht eine kaum übersehbare Zahl an
Bibliotheksentwürfen und -schriften, in denen nach einer
architektonischen Form gesucht wird, die der Expansion der
Gutenberggalaxis, der Erweiterung des Nutzerkreises und der räumlichen
Ausdifferenzierung der Bibliotheksfunktionen gerecht wird. Ausgehend von
der öffentlichen Bibliotheksdiskussion, wie sie auch und besonders in
der Architekturpresse geführt wird, untersucht der Beitrag am Beispiel
der französischen *Bibliothèque royale* den Entwurf der modernen
Bibliothek als Zentral- bzw. Rundbau mit einem Lesesaal in seiner Mitte.
In diesen Entwurf schreiben sich in der Zeitspanne von 1790 bis 1840
vollkommen verschiedene Wissens- und Körperordnungen ein. Ihre
Verbindung besteht dort, wo das absolute räumliche Zentrum der
Bibliothek mit einem allsehenden Auge besetzt zu sein scheint.}

\vspace{5mm}

\import{../05wagner/}{master.tex}

%%%% hobohm

\authortoc{Hans-Christoph Hobohm}
\chapter*[Vom Ort zum Akteur]{Vom Ort zum Akteur. Heterotopologie + Akteur-Network-Theorie auf die Bibliothek bezogen}
\addcontentsline{toc}{chapter}{Vom Ort zum Akteur. Heterotopologie + Akteur-Network-Theorie auf die Bibliothek bezogen}
\refstepcounter{chapter}
\begin{flushright}
{\large Hans-Christoph Hobohm}
\end{flushright}

\vspace{5mm}

\import{../06hobohm/}{master.tex}

%%%% Verzeichnis der Autorinnen und Autoren

\chapter*[Verzeichnis der Autorinnen und Autoren]{Verzeichnis der Autorinnen und Autoren}
\addcontentsline{toc}{chapter}{Verzeichnis der Autorinnen und Autoren}
\refstepcounter{chapter}

\noindent \textbf{Karin Aleksander}, Dr. phil., ist Leiterin der Bibliothek des Zentrum für transdisziplinäre Geschlechterstudien, Humboldt Universität zu Berlin.

\vspace{5mm}

\noindent \textbf{Ute Engelkenmeier}: Ausgestattet mit einem ausgiebigen Studium des
Bibliothekswesens (FHBD Köln, Musikbibliothekarisches Zusatzstudium HBI
Stuttgart) und der Bibliotheks- und Informationswissenschaft (Humboldt
Universität zu Berlin) seit 1995 in der Universitätsbibliothek Dortmund
tätig, dort zehn Jahre lang Öffentlichkeitsarbeiterin und seit vier
Jahren Geschäftsbereichsleitung Medienbereitstellung und Information
(Benutzung).
Interessensschwerpunkte: Nutzerforschung, Lernortentwicklung, Marketing
und Management
Kontakt: ute.engelkenmeier@tu-dortmund.de

\vspace{5mm}

\noindent \textbf{Frank Hartmann} ist Professor für Geschichte und Theorie der Visuellen Kommunikation an der Bauhaus-Universität Weimar. Er hat zahlreiche Publikationen zu Medienphilosophie verfasst und ist Herausgeber der Schriftenreihe \enquote{Forschung Visuelle Kultur}, darunter der Band \enquote{Vom Buch zur Datenbank. Paul Otlets Utopie der Wissensvisualisierung} (Berlin 2012).

\vspace{5mm}

\noindent \textbf{Hans-Christoph Hobohm} ist seit 1995 Professor für Biblio\-theks- und Informationswissenschaft an der Fachhochschule Potsdam. Neben Wissensmanagement und Informationsverhaltensforschung ist eines seiner Lehr- und Forschungsgebiete Bibliothekstypologie und Bibliothekstheorie.

\vspace{5mm}

\noindent \textbf{Kirsten Wagner} ist Professorin für Kulturwissenschaft und
Kommunikationswissenschaft am Fachbereich Gestaltung der Fachhochschule
Bielefeld. Davor war sie wissenschaftliche Mitarbeiterin am
Kulturwissenschaftlichen Institut der Humboldt-Universität zu Berlin, wo
sie im Rahmen des Sonderforschungsbereiches 447 „Kulturen des
Perfomativen“ zu historischen und gegenwärtigen Formen räumlicher
Wissensorganisation gearbeitet hat, darunter auch zur Bibliothek als
materiellem Raum des Wissens. Ihre aktuellen Arbeits- und
Forschungsschwerpunkte umfassen neben Fragen der Wissensorganisation
Raum- und Wahrnehmungstheorien sowie eine Anthropologie der Architektur.

\cleartoverso

\cleartorecto


\end{document}