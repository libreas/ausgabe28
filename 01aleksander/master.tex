\noindent Seit Dekaden schon wandert das Bild vom notwendigen Wandel der
Bibliotheken als der einzigen Konstante für ihre weitere Existenz durch
die Medien und die Köpfe. Wer die Kombination Wandel und Bibliothek in
eine Suchmaschine tippt, bekommt mehr als 320.000 Einträge. Dabei geht
es vor allem um neue Bibliotheksbauten, um Digitalisierung und E-Books
sowie um virtuelle und Gruppenarbeitsplätze. Auf diesen Gebieten gab und
gibt es immer noch Nachholbedarf in vielen Regionen. Aber ist damit
alles erfasst, was mit Wandel gemeint sein könnte? Was soll/muss eine
Bibliothek in einer Wissensgesellschaft idealerweise leisten? Mögliche
Antworten auf diese Frage treffen den Kern bibliothekarischer Arbeit.
Warum wird was wie und wozu gesammelt? Ist das Sammeln mit der
Formalerfassung beendet? Wie wird das Sammelgut aufbereitet und
erschlossen? Was ist eine ideale Bibliothek? Auf jeden Fall stimmt:
\enquote{Die perfekte Bibliothek ist \ldots{} Weg und Ziel zugleich.}
(Keller 2009) Die Zeitschrift LIBREAS hatte diese Perspektive in den
bisherigen zehn Jahren ihrer Existenz unter den verschiedensten Aspekten
im Auge.

Nebenbei: Dieser zehnte respektive 10. Jahrestag passt zahlenmäßig so
besonders für diese Open Access-Zeitschrift, weil er aus 1 und 0
komponiert ist, weil er sofort an Bits und Internet erinnert oder an 1,0
für eine Top-Leistung. Es passt gar nicht im Sinne von 1.0 als Code für
die erste industrielle Revolution. Ich will damit keine Analogie für die
Bibliotheken andeuten, sie sind vor allem nicht primär mit industriellen
oder ökonomischen Maßstäben zu messen! Aber selbst in dieser Zeit der
Industrie 1.0 gab es in der Bibliotheksgeschichte revolutionierende
Ereignisse, die bis heute wirken. Bekannt ist sicher der
Paradigmenwechsel gegenüber der damaligen systematischen Ordnung, indem
die Enzyklopädisten in England und Frankreich das \enquote{Wissen
alphabetisch ordneten} und \enquote{mit Querverweisen vom üblichen Kreis
des Wissens zu einem ‚Netz des Wissens` übergingen}. (Capurro 2001)

Heute, wo mit Wörtern wie \enquote{Arbeit 4.0} oder \enquote{Bibliothek
4.0} (Hobohm 2007) phantasiert wird, sind die Bibliotheken
herausgefordert, dieses Netz des Wissens sowohl immer enger zu knüpfen
als auch gleichzeitig weiter zu spannen, damit die publizierte
Information immer besser die Interessen der Nutzenden leiten kann. Das
Bild der Bibliothek als Wissensspeicher, als Ort des Lernens ist für den
Platz von Bibliotheken in der Wissensgesellschaft zentral. Betrachten
wir die Bibliothek nicht als Raum oder Ort, sondern entsprechend ihrer
Funktionen, dann wird an erster Stelle meist der Dreiklang von Sammeln,
Ordnen und zur Verfügung stellen genannt (Ewert / Umstätter 1997). Aber
reicht das heute noch aus? Die drei Funktionen erfüllen auch ein Verlag,
der Buchhandel oder ein Internetantiquariat. Die Bibliothek kann und
muss mehr aus ihrem Potential herausholen, wenn sie den kritischen
Stimmen, die ihr Ende voraussagen, entgegentreten will. Ihr Potential
sind die Ergebnisse wissenschaftlicher Arbeit der verschiedensten
Disziplinen. Ihre zentrale Aufgabe muss sein, diese Ergebnisse für
Studium, Wissenschaft und Praxis so zu erschließen, dass sie effektiv
genutzt werden können und unmittelbar wieder neue wissenschaftliche
Fragen, also Unwissen, produzieren.

Meine Frage nach einer transdisziplinären Bibliothek soll den ständig
notwendigen Wandel in Bibliotheken erneut zuspitzen auf die Frage
Formal- versus Inhaltserschließung. \emph{Kann die Bibliothek unter
diesem Aspekt als transdisziplinäre gedacht werden?}

Bisher tauchte die Charakteristik transdisziplinäre Bibliothek noch
nicht explizit in den 27 LIBREAS-Ausgaben auf. Allerdings waren
\enquote{Philosophische Fragen in Bibliothek und
Bibliothekswissenschaft} das Thema im Winter 2006. Im Editorial schrieb
die Redaktion damals zur Institution Bibliothek als Bezugssystem der
Bibliothekswissenschaft: Dieses Bezugssystem \enquote{existiert nicht
unabhängig von anderen Systemen, sondern \ldots{} vielmehr mit diesen in
Wechselwirkung -- ein Faktum, welchem unter den Stichworten ‚Inter-`
bzw. ‚Transdisziplinarität` seit der zweiten Hälfte des 20. Jahrhunderts
verstärkt Rechnung getragen wird.} (Redaktion LIBREAS 2006) Ganz frisch
publiziert, sichtet Hans-Christoph Hobohm die Servicewissenschaft als
transdisziplinär und Teil der Informationsswissenschaft. (Hobohm 2015)
Ich selbst folge dem Thema Inter-/Transdisziplinarität seit der Gründung
unseres Zentrums für interdisziplinäre Frauenforschung (ZiF, 1989) und
verstärkt seit seiner Umbenennung in Zentrum für transdisziplinäre
Geschlechterstudien (ZtG, 2003). (Aleksander 2004, 2005)

In Fachdatenbanken fand ich zwar einzelne Titel bei trunkierter
Kombination beider Wörter, aber keine Angaben zur Kombination
\enquote{transdisziplinäre Bibliothek}. Danach fragte ich eine
Suchmaschine und bekam einen einzigen Beitrag! Er zeigte die wirklich so
bezeichnete \enquote{transdisziplinäre Bibliothek} von Robert Jungk
(1913-1994), seine \enquote{Zukunftsbibliothek} in Salzburg, die sich
seit 1985/86 in 41 Themenbereichen mit nachhaltigen Zukunftsfragen
beschäftigt.\footnote{Die angegebenen Gründungsdaten widersprechen sich:
  auf der Online-Seite der Bibliothek bei der Angabe der Team-Zeiten
  wird 1985 genannt, auf der Seite über Robert Jungk steht, dass
  \enquote{1986 ein lang gehegter Wunsch \ldots{} in Erfüllung ging}
  (vgl.: \url{http://www.jungk-bibliothek.at/} und
  \url{http://www.robertjungk.at/jungk_seinebibliothek.htm}).}

Ähnlich, mit ausgewählten gesellschaftlichen Problemfeldern, beschäftigt
sich seit 1979/2004 die Bibliothek der Fakultät für Interdisziplinäre
Forschung und Fortbildung (IFF) an der Alpen-Adria-Universität
Klagenfurt/Wien/Graz. Die Bibliothek in Wien versorgt ihre Fakultät mit
Literatur zu \enquote{Fragen nach dem Umgang mit ‚public goods` wie
Gesundheit, Umwelt, Raum, Technologie, Bildung oder
Wissenschaft}.\footnote{\url{http://www.uni-klu.ac.at/iff/inhalt/253.htm}.}
Ein grundlegendes Arbeitsprinzip der Fakultät ist es deshalb
\enquote{methodische und disziplinäre Vielfalt} anzuregen, die durch
interdisziplinär zusammengesetzte Teams und Transdisziplinarität im
Sinne von \enquote{Kooperation mit Personen und Organisationen \ldots{},
die mit dem jeweiligen Problemfeld befaßt sind}. (ebenda) gewährleistet
wird.

Und ähnlich beschäftigt sich auch die Genderbibliothek des Zentrum für
transdisziplinäre Geschlechterforschung seit 1990, also seit 25 Jahren,
mit den verschiedenen Aspekten der Geschlechterforschung. Wir
kooperieren in unserem Studiengang (gegründet 1997) zurzeit mit
Lehrangeboten aus 19 verschiedenen Fächern. Dafür stellen wir Literatur
aus 18 verschiedenen Themenkomplexen zur Verfügung, das sind zurzeit
circa 63.000 Datensätze im Online-Katalog GReTA.\footnote{\url{http://genderbibliothek.de/}.}

Es gibt sie also schon, die transdisziplinäre Bibliothek. Es ist die
Frage, warum in Spezialbibliotheken möglich ist, was in großen
Universitätsbibliotheken nicht gehen soll?

In dem gerade publizierten \enquote{Horizon Report \textgreater{}Library
Edition}\footnote{\url{http://cdn.nmc.org/media/2015-nmc-horizon-report-library-EN.pdf}.}
steht als eines von zwei Langzeit-Trends für wissenschaftliche
Bibliotheken: \enquote{increasing the accessibility of research
content}. Dieser Ausdruck kann nur den technischen Zugang zum Inhalt der
Forschung meinen, aber das Verb \enquote{to make accessible} bedeutet
auch \enquote{erschließen}, worauf ich mich beziehen möchte. Mich
interessiert primär die inhaltliche Verschlagwortung oder Indexierung,
weil sie die Inhalte der einzelnen Disziplinen zum Gegenstand hat und
damit das inhaltliche Potential der Bibliothek bildet. Die drei
erwähnten transdisziplinären Bibliotheken sehen genau diese Arbeit als
wichtigsten Service für ihre Nutzer\_innen:

\begin{itemize}
\item
  In der Robert-Jungk-Zukunftsbibliothek wirbt das Vier-Personen-Team (=
  3 ½ Stellen) auf der Homepage mit den selbst verfassten Abstracts für
  die über 14.000 Medien im Bestand.\footnote{\url{http://www.jungk-bibliothek.at/}.}
\item
  Die IFF-Bibliothek in Wien mit zwei Personen (ca. 1 ½ Stellen)
  verschlagwortet nach der Gemeinsamen Normdatei (GND), aber mit eigener
  Klassifikation. In einer Umfrage im Jahr 2000 forderten die meisten
  Lehrenden der Fakultät eine verbesserte Schlagwortsuche
  beziehungsweise eine Indexsuche über Inhalte von Büchern und
  Zeitschriften. (Lube 2000)
\item
  Und in der Genderbibliothek, die eine One Person-Library ist (mit
  einer studentischen Mitarbeiterin), verschlagworte ich bis zur
  Artikelebene nach eigener Indexliste mit Blick auf die Begriffe der
  Geschlechterforschung und ihrer interferierenden Gebiete.
\end{itemize}

In vielen traditionellen Bibliotheken divergieren die unterschiedliche
Aufgaben und Erwartungen in Bezug auf Formal- und Inhaltserschließung
zwischen Leser\_innen und Bibliothek. (Weinheimer 2014)\footnote{Zum
  Illustrieren sehen Sie folgenden Trickfilm über eine simple
  Buchausleihe: \url{https://www.youtube.com/watch?v=C7M69J7IwyQ}.} Auch
wenn vor Jahren behauptet wurde, die Diskussion \enquote{Holding versus
Access} sei längst beendet und zugunsten eines \enquote{Access to
Information} entschieden (Ball 2008), so scheint mir der Aspekt
Inhaltserschließung immer noch zu selten berücksichtigt. Die inhaltliche
Arbeit ist nach wie vor von der Bibliotheksseite her meist zweitrangig
gegenüber der Formalerschließung oder muss vernachlässigt werden, weil
intellektuelle Arbeit aus wirtschaftlicher Sicht zu teuer ist.

Das zeigte sich schon beim Umstieg auf AACR2 vor über zehn Jahren und
zeigt sich nun wieder beim Übergang zu RDA seit Oktober 2015. Die
Deutsche Nationalbibliothek (DNB) hält zwar prinzipiell an
Klassifikationen und kontrollierten Vokabularen fest, die sich über
viele Jahrzehnte als Mittel der Wissensorganisation und
Informationsverdichtung bewährt haben, auch, weil sie Eigenschaften
aufweisen, die sie für Web-Anwendungen nützlich machen. (Junger 2015)

Ganz bewusst konzentrieren sich die Verantwortlichen in den
Expertengruppen (EG) der DNB hier mit ihrem bisher entwickelten
Regelwerk auf die Formalerfassung. Zur inhaltlichen Erschließung bemerkt
die DNB auf ihrer Homepage: \enquote{Obgleich die RDA eigene Kapitel für
die Inhaltserschließung vorsehen, ist auf absehbare Zeit mit der
Ausgestaltung dieser Kapitel nicht zu rechnen. Das bisher im
deutschsprachigen Raum verwendete Regelwerk zur verbalen
Sacherschließung RSWK soll daher überarbeitet und dabei deutlich
verschlankt werden.}\footnote{\url{http://www.dnb.de/DE/Standardisierung/standardisierung_node.html}.}
Im August 2015 schreibt Karin Schmidgall im ASPB-Blog aus der
Arbeitsgemeinschaft RDA: \enquote{Wie die Sacherschließung erfolgen
soll, wird noch in den zuständigen Gremien besprochen.}\footnote{\url{http://www.aspb.de/rda/}.}
Aus dem Protokoll der EG Sacherschließung (früher RSWK-Anwenderkreis)
vom 14. April 2015 wird klar, dass

\begin{quote}
\enquote{dem Standardisierungsausschuss deutlich geworden (ist), dass
eine Neukonzeption des Regelwerks sehr aufwendig werden würde. Deshalb
habe man beschlossen, zunächst die RSWK bis Oktober 2015 auf einen Stand
bringen zu lassen, der der Inhaltserschließung mit dem Umstieg auf RDA
ein funktionierendes Arbeiten mit den RSWK ermögliche, sodass sowohl für
Formal- wie Inhaltserschließung aktuelle Regelwerke vorliegen. Was die
Neukonzeption der Inhaltserschließung angehe, so sei eine strategische
Gruppe gegründet worden, die sich dieser annehmen werde.}\footnote{\url{https://www.hbz-nrw.de/dokumentencenter/produkte/verbunddatenbank/aktuell/rswk_anwenderkreis/protokoll_eg_inhaltserschliessung_2015.pdf},
  S.\,11.}
\end{quote}

Es wird also dauern! In der Zwischenzeit bleibt -- leider -- aktuell,
was aus der Sicht der Geschlechterforschung kein ausreichender Service
ist: die Nichtbeachtung wichtiger Grundbegriffe oder ihrer Beziehungen
in der aktuell gültigen Gemeinsamen Normdatei. Solange diese Zustände
existieren, werden Forschende und Studierende der Geschlechterstudien in
den traditionellen Universitätsbibliotheken nicht die Werke finden, die
sie benötigen, obwohl sie meist sogar vorhanden sind.\footnote{Dass sie
  vorhanden sind, ist eine Folge der ca. 25 Studiengänge für
  Geschlechterstudien/Gender und ca. 130 Professuren an Universitäten
  sowie 44 an Fachhochschulen (das sind seit 2000 unverändert zwischen
  0,4 und 0,5 Prozent aller Professuren in der BRD). Quelle:
  \url{http://www.zefg.fu-berlin.de/Datenbanken/index.html}.} Mehrere
Beispiele für diese Praxis habe ich ausführlicher in dem LIBREAS-Artikel
\enquote{Die Frau im Bibliothekskatalog} beschrieben. (Aleksander 2014)
Sie illustrieren, womit zum Beispiel die Geschlechterstudien mit ihrem
inter-/ transdisziplinären Charakter beim Recherchieren konfrontiert
sind:

\begin{enumerate}
\def\labelenumi{\arabic{enumi}.}
\item
  Der Vergleich zweier wissenschaftlicher Begriffe zeigt, dass der
  Fachbegriff \enquote{Geschlechterverhältnisse} aus den Gender Studies
  gegenüber dem Fachbegriff \enquote{Higgs-Teilchen} aus der Physik
  nicht nur später als dieser, nicht nur nicht als Sachbegriff, sondern
  auch unkorrekt -- als Synonym für \enquote{Geschlechtsverhältnis}
  beziehungsweise \enquote{Geschlechterbeziehung} -- verwendet wird,
  obwohl er in einem Lexikon in der Liste der fachlichen
  Nachschlagewerke für die GND seit 1994 enthalten ist.
\item
  Ein Beispiel, wo der traditionelle oder männliche Blick regiert, ist
  die folgende hierarchische Anordnung der Sachbegriffe: Sozialpolitik,
  Frauenpolitik, Gleichstellungspolitik. Einen Sachbegriff
  \enquote{Männerpolitik} gibt es nicht, ebenso wenig
  \enquote{Geschlechterpolitik}!
\item
  Ein weiteres Beispiel zeigt, dass der Rechercheaufwand unökonomisch
  wird, weil eine RSWK-Regel bestimmt, dass die männliche und weibliche
  Form zur Bezeichnung desselben Gegenstandes nur verwendet werden darf,
  wenn es sich um einen Vergleich handelt. Deshalb sind zum Beispiel
  jene, die zum Thema \enquote{Frauen an Universitäten} forschen,
  gezwungen, alle Titel unter dem Schlagwort \enquote{Gelehrter}
  durchzuforsten (in diesem Fall über 3.600 Titel!), wenn sie etwas mehr
  über weibliche Gelehrte erfahren möchten (das Schlagwort wird
  lediglich für zwölf Titel ausgewiesen, in denen es nur um Frauen
  geht)!\footnote{Der Buchumschlag zur Geschichte von Gelehrten an der
    Universität Innsbruck zeigt 10 Männer und zwei Frauen. Das Buch wird
    mit dem Schlagwort \enquote{Gelehrter} angesetzt: Töchterle,
    Karlheinz (Hrsg.): Köpfe zwischen Krise und Karriere. Innsbruck :
    Innsbruck Univ. Press, 2010.}
\item
  Und letztlich ein Beispiel, wo der Titel nicht gefunden wird, obwohl
  das Buch vorhanden ist: Das passiert, wenn ein Sammelband wie
  \enquote{Rechtsextremismus und Gender}\footnote{Birsl, Ursula:
    Rechtsextremismus und Gender. Opladen : Budrich, 2011.} mit den
  Schlagworten \enquote{Rechtsradikalismus} und
  \enquote{Geschlechterforschung} verschlagwortet wird. Ein Blick ins
  Inhaltsverzeichnis erhellt: In dem Sammelband finden sich von 16
  Titeln acht, die sich mit Rechtsextremismus und Männern beschäftigen
  und sieben zu Frauen und Mädchen. Bei einer Recherche zu
  \enquote{Rechtsradikalismus} beziehungsweise
  \enquote{Rechtsextremismus} und \enquote{Frau} oder \enquote{Mann}
  wird dieser Sammelband aber nicht angezeigt.
\end{enumerate}

Das waren Beispiele für die intellektuelle Verschlagwortung. Andere
Fachgebiete können hier sicher ergänzen.

Und ein zweiter Zustand bleibt ebenfalls bis auf weiteres erhalten: die
GND ist nämlich auch die Grundlage für die maschinelle oder
softwaregestützte Erschließung, deren Projekt die DNB sinnigerweise
\enquote{PETRUS} getauft hatte und das unter anderem bereits für
Netzpublikationen erfolgreich eingesetzt wird. Das \enquote{erfolgreich}
bezieht sich auf die Testergebnisse, die laut Uhlmann \enquote{durch die
jeweils für die Fachgebiete zuständigen Mitarbeiter der Abteilung
Inhaltserschließung der DNB} bewertet wurden. (Uhlmann 2013, S.\,29)
Uhlmann schlussfolgert richtig:

\begin{quote}
\enquote{Die Grenzen einer automatischen Beschlagwortung beginnen dort,
wo an die inhaltliche Erschließung der Anspruch gestellt wird, eine
möglichst eindeutige, d.h. spezifische und nicht redundante, Essenz
eines Textes zu formulieren. \ldots{} Es muss daher gar nicht der
Versuch unternommen werden, beide Erschließungsformen mit demselben
Maßstab zu messen, {[}\ldots{}{]} Automatische Beschlagwortung ist immer
abhängig von dem zugrunde liegenden Text und der zur Beschreibung
genutzten Terminologie, also von den Begriffen, die vorhanden sind oder
eben auch nicht.} (Uhlmann 2013, S.\,34f.)
\end{quote}

Wie können die speziellen transdisziplinären Bibliotheken Informationen
für bestimmte Nutzungsgruppen effektiver erfassen und recherchieren?

Es geht dabei zuerst um die Inhalte der Frage, um das
In-Beziehung-Setzen verschiedener Sichtweisen für einen bestimmten
Zweck. Hier fällt vielleicht auf, dass ich bisher noch nicht versuchte,
das Wörtchen \enquote{transdisziplinär} zu definieren. Genau wie ein
Team der erwähnten Fakultät für Interdisziplinäre Forschung und
Fortbildung (IFF) aus Wien in einem aktuellen Buch schreibt, möchte ich
den existierenden 20 Definitionen nicht die 21. hinzufügen, \enquote{und
das aus Gründen, die selbst wiederum mit inter- und transdisziplinärer
Methodik zusammenhängen}, wie sie betonen. \enquote{Weniger die Begriffe
und damit auch die theoretischen Hintergründe sind zwischen inter- und
transdisziplinär Kooperierenden zu vergemeinschaften, sondern die
Probleme, Anliegen und Fragestellungen. Und dies passiert immer
{[}\ldots{}{]} im ganz konkreten gemeinsamen Tun.} (Dressel u.a. 2014,
S.\,300)

Für die Bibliothek, speziell beim Indexieren, brauchen wir also die
nicht vergemeinschafteten Begriffe oder Wörter, also die disziplinären,
aber mit Blick von der einen durch die andere auf die jeweilige
Disziplin oder Aufgabe. Deutlich wird das zum Beispiel am Begriff
\enquote{Bewegung}. Er wird in mehr als einem Dutzend verschiedener
Disziplinen verwendet. Beim inter- und transdisziplinären Tun empfehle
ich einer Sportwissenschaftlerin möglicherweise auch Titel aus der
Medizin oder Geschichte und einem Soziologen auch die aus Physik oder
Biologie. Die Verknüpfungen entstehen dann (hoffentlich) in den
individuellen Köpfen entsprechend der Frage. Auf diese Weise entdecken
die Forschenden Neues, das sie beim Gang in die Bibliothek nicht gesucht
haben. Sie fühlten ihr eigenes Nichtwissen auf einem Gebiet und
vernetzten bekannte mit neuen Ideen. Auch für mich als Bibliothekarin
sind solche Momente immer wieder Aha-Erlebnisse der besonderen Art.
Solche Erlebnisse sind bei uns möglich, weil wir die Inhalte der meisten
Publikationen \enquote{kennen}, und weil wir mit unseren selbst
vergebenen Schlagworten die Titel aus dem eigenen Netz der Inhalte
herausfischen können, aus dem eigenen Kopf und dem eigenen Katalog, ohne
disziplinäre Grenzen. Obwohl die Recall- und Precision-Rate unseres
GReTA-Katalogs immer besser wird, frage ich mich, wie diese individuelle
Fähigkeit weiter zu \enquote{verobjektivieren} ist, also in das
Rechercheprogramm zu implementieren.

Um solche Erlebnisse \enquote{produzieren} zu können, ist sicher eine
andere Betrachtungsweise über die Wissensproduktion notwendig. Ihre
Ergebnisse werden in Bibliotheken schon jahrhundertelang nach dem
linearen und hierarchischen Disziplinenprinzip der Universitäten
aufgestellt und verschlagwortet. Inter- und Transdisziplinarität sind
hier, wie Mittelstrass frohlockte, \enquote{Stachel im Fleisch unserer
institutionellen wissenschaftlichen Ordnung}, die \enquote{flüssig
werden} müsse. (Mittelstrass 2012, S.\,12) Das scheint mir ein gutes
Bild! Alle einzelnen Disziplinen tragen das inter- und transdisziplinäre
Potential in sich. Je stärker sie sich verfestigen, können Fragen in
ihrem disziplinären Rahmen nicht mehr gelöst werden. Flüssigmachen kann
diese alte Ordnung vor allem das Denken selbst. Wenn es sich seiner
eigenen Beschränktheiten wieder bewusst wird.

Was ist damit gemeint? Das lässt sich zum Beispiel an der Entwicklung
vom geozentrischen zum heliozentrischen Weltbild erläutern. Heute
vertreten nur sehr wenige Personen das geozentrische Weltbild, dass sich
die Sonne um die Erde im Mittelpunkt dreht. Den meisten ist als Ergebnis
ihrer Bildungssozialisation rational klar, dass sich die Erde um die
Sonne dreht und es weitere Sonnensysteme gibt. Wir zweifeln nicht daran,
obwohl der alltägliche Schein uns weiterhin etwas anderes vorgaukelt.
Wir haben die Kopernikanische Wende für dieses Gebiet in unserem
Wissenssystem als \enquote{richtiges Wissen} verankert. Aber 100 hundert
Jahre nach Einsteins Relativitätstheorie besteht die moderne Auffassung
darin, dass wegen der Relativität kein Bezugspunkt im Raum herausgehoben
werden kann -- und wir diese Unterscheidung gar nicht mehr brauchen!
Dieser inhaltliche Wandel der Weltbilder war historisch immer mit
starken Gefühlen verbunden. Welche Gefühle, welche Denktradition hindert
uns, eine Inhaltserschließung transdisziplinär zu organisieren? Brauchen
wir eine Kopernikanische Wende für den Aspekt Formal- versus
Inhaltserschließung und fühlen uns erst \enquote{ganz}, wenn beide
gleichberechtigt sind?

Möglicherweise verkörperte der frühe Professorenbibliothekar noch diese
Einheit für seine Disziplin(en). Ist das unbewusst transdisziplinär?
Diese Einheit begann sich zu spalten mit dem Anwachsen des Bestandes,
der dann trotz innovativem Schlagwortkatalog vorrangig formalisiert und
verwaltet werden musste. Damit wandelte sich auch die Arbeitsteilung in
Bibliotheken. Erstmals wurden Frauen eingestellt, später übernahm die
Technik das Zepter. Heute, wo die Informationstechnik das Potential hat,
auch die Inhalte abzubilden und zu ordnen, stecken wir in der
Formalisierung fest. Denn die Inhaltserfassung geht (noch) nicht ohne
das menschliche Denken. Ist ‚automatic indexing` immer noch ein
Oxymoron?\footnote{Zum automatischen Indexieren zitiert Dieter E. Zimmer
  (2000) in seinem Buch \enquote{Die Bibliothek der Zukunft} eine
  Wertung von Hans H. Wellisch von 1991: \enquote{Es ist \ldots{} nicht
  unvernünftig, ‚automatic indexing` für ein Oxymoron zu halten.}} Eine
neue Denk- und Handlungsweise, die Wissensbestände inhalts- und
nutzungsadäquat zu erfassen und zur Verfügung zu stellen, ist meines
Erachtens die transdiziplinäre Inhaltserschließung in den Bibliotheken.

Vielleicht irritiert an dieser Kombination von Transdisziplinarität und
Bibliothek, dass die Begriffe Inter- wie auch Transdisziplinarität meist
nur für die Forschung benutzt werden. Es sind aber keine Theorien,
sondern Methoden. Ebenso wie Georg Ruppelt an die Adresse von
Bibliothekswissenschaftler\_innen fragte \enquote{Warum lassen sie sich
von anderen Wissenschaften vorgeben, was Wissenschaft ist (Was ist sie
denn?)} (Ruppelt 2005, S.\,8), gilt die Frage: Wer schreibt vor, wo
transdisziplinäres Arbeiten möglich oder erlaubt ist? Entscheidend muss
der Nutzen sein. Bei all den zahlreichen Disputen um Inter- und
Transdisziplinarität wird als ein beide unterscheidendes Merkmal meist
der unmittelbare Praxisbezug bei transdisziplinärer Zusammenarbeit
herausgehoben. Auch Wissenschaft selbst kann als eine Form von Praxis
betrachtet werden; es kommt wie bei Kopernikus auf den Bezugspunkt an.
Transdisziplinarität leistet eine doppelte Grenzüberschreitung:

\begin{quote}
\enquote{Die integrative Kraft kann erstens wissenschaftsintern
konstruiert sein, dann wird nach der verbindenden Rationalität in den
historisch entwickelten unterschiedlichen Rationalitätsformen gesucht.
Sie kann aber -- zweitens -- gleichzeitig darauf gerichtet sein, die
Wissenschaft als systemischen Zusammenhang mit der Lebenswelt in
Verbindung zu bringen. Beide Bedeutungen von Transdisziplinarität muss
man sich {[}\ldots{}{]} in einem Ergänzungsverhältnis denken.}
(Mückenberger/Timpf 2010, S.\,205)
\end{quote}

Bibliotheken stehen genau an der Schnittstelle zwischen Wissenschaft und
Lebenswelt! \enquote{`Trans` steht {[}\ldots{}{]} für die Überschreitung
der Grenze zwischen Wissenschaft und Nichtwissenschaft, {[}\ldots{}{]}
für eine deutliche Anwendungsorientierung} (Hark 2003, S.\,83) oder für
gemeinsame Instrumentenentwicklung sowie die anerkannte Gleichwertigkeit
wissenschaftlichen und nicht-wissenschaftlichen Wissens.\footnote{Vgl.
  auch ein Modell von Wissenschaft und Praxis in klassischen, inter- und
  transdisziplinären Austauschprozessen bei Heintel 2009, S.\,191.} Die
Basis für den Wandel in der Inhaltserschließung von Bibliotheken sind
dabei nicht nur die speziellen Termini der Disziplinen, ihre
Verschränkungen und eine neue Wissensorganisation, sondern darüber
hinaus vor allem allgemeine Denkprinzipien, die \enquote{Verschiebung
der Optiken} (Barad 2015, S.\,177) wie zum Beispiel: die Abkehr von einer
monodisziplinären, androzentrischen, eurozentrischen und auch
anthropozentrischen Sichtweise. In diesem Sinne urteilte Elizabeth
Minnich schon 1982 für Feministinnen: \enquote{Was wir tun, ist
vergleichbar mit Kopernikus, der unseren Geozentrismus erschütterte, mit
Darwin, der unseren Artenzentrismus erschütterte. Wir erschüttern den
Androzentrismus, und dieser Wandel ist ebenso fundamental, wie
gefährlich und spannend.}\footnote{Original: \enquote{What we are doing
  is comparable to Copernicus shattering our geo-centricity, Darwin
  shattering our species-centricity. We are shattering andro-centricity,
  and the change is as fundamental, as dangerous, as exiting.}
  (Übersetzung im Text von K.A.).} (Minnich 1982, S.\,311) Das bedeutet:
Wer aus einer monodisziplinären Sicht aussteigt, erschließt sich multi-,
inter- und transdisziplinäre Einsichten und Praktiken. Mit Lichtenberg
könnten wir lästern: Wer nur Chemie versteht, versteht auch diese nicht.

Die androzentrische Sichtweise ist schwerer zu überwinden, weil sie in
das Wissen, in das Wissenschaftssystem wie in unser Alltagsleben
eingeschrieben ist. Hier bewies die inter-/ nationale Frauen- und
Geschlechterforschung im Laufe ihrer über fünfzigjährigen Geschichte,
wie männlich die Wissenschaft ist und lieferte neue Interpretationen zu
alten Zusammenhängen.

In der internationalen Wissenschaftskommunikation offenbart sich immer
mehr, wie ein eurozentrischer Standpunkt Zusammenarbeit behindert. Ganz
deutlich zeigt sich das an den Diskussionen, was überhaupt und wie
wichtig eine Globalgeschichte ist. Und schließlich fordert eine
Weltsicht, die alle uns Menschen umgebenden Dinge, Gegenstände, Tiere
oder: Aktanten, Apparate und Artefakte als netzwerkartig miteinander
verbunden denkt, sich von der Auffassung zu verabschieden, dass der
Mensch der Mittelpunkt der Welt sei.

In diesem Sinne schreibt ein interdisziplinäres Team im Vorwort zum Buch
\enquote{Geschlechter Interferenzen}:

\begin{quote}
\enquote{D.h. es geht auch ganz wesentlich um eine Krise der
Wissensformen, ‚klassischer` wissenschaftlicher Bestimmungen,
Messbarkeitsannahmen und Klassifikationen, die nicht mehr in der Lage
sind, die materielle, intraaktive Dynamik der ‚Krisen` zu begreifen. Die
Welt, ihre institutionellen Verfestigungen, ihre bestimmbare Relevanzen
und Unterschiede produzierenden Apparate transformieren sich, werden
instabil und -- vorübergehend und partiell -- unbestimmt, lösen sich auf
-- und rekonfigurieren sich.} (Bath u.a., 2013, S.\,17f.)
\end{quote}

Im Ergebnis einer zu verändernden Weltsicht kommen wir dann zu einer
transdisziplinären, agentisch-realistischen Betrachtungsweise, zu einer
neuen Ethico-Onto-Epistemologie, wie sie die US-Forscherin Karen Barad
seit einigen Jahren entwickelt (Barad 2012, 2013, 2015).

Zum Verständnis dieser im aktuellen Feminismus diskutierten Auffassung
führt Barad als Teilchenphysikerin zurück zu einem Experiment in der
Physikgeschichte. Diesmal geht es nicht um Kopernikus, sondern um
Kopenhagen, um die Kopenhagener Deutung, um Bohr und Heisenberg, also um
den Welle-Teilchen-Dualismus in der Quantenphysik. Licht hat im
physikalischen Experiment einen dualen Charakter, es kann Welle oder
Teilchen sein. Bei der Kopenhagener Deutung ist nun wichtig zu
entscheiden, ob dieser Dualismus dem Licht als materieller Erscheinung
wesentlich inhärent ist oder ob diese nicht klar zu bestimmende
Unschärfe ein Ergebnis des menschlichen Messprozesses ist. Karen Barad
erwähnt in einer Fußnote, dass beide Physiker 1924 heiß über die Deutung
diskutierten und schließlich Heisenberg gegenüber Bohr im Nachwort
seiner Schrift eingestand, dass er wesentliche Punkte übersehen hätte.
Gelehrt wird heute die Heisenbergsche Unschärferelation, wonach Impuls
und Ort nicht gleichzeitig messbar seien. Dabei beweisen heutige
Experimente eher Bohrs Ansicht, dass \enquote{Welle} und
\enquote{Teilchen} klassische Beschreibungen sind, \enquote{die auf
verschiedene, einander wechselseitig ausschließende Phänomene referieren
und nicht auf unabhängige, physikalische Objekte.} (Barad 2015, S.\,41f.)
Soll heißen: die klassischen Begriffe sind auf diese Phänomene gar nicht
anwendbar, weil sie ab einer bestimmten Grenze nicht unterscheidbar
sind. Damit liefert Bohrs \enquote{Physik-Philosophie}, wie Barad sie
bezeichnet, wertvolle Hinweise auf den Prozess, wie wir Wissen
produzieren. Dieser Prozess ist nie auf eine Disziplin beschränkt,
sondern immer schon verstrickt mit anderen Annahmen, verstrickt, wie
Barad (2012, S.\,19f.) das nennt.

Wenn wir in Bibliotheken mehr darüber wissen, wie Wissen produziert
wird, liefert das möglicherweise Analogien dafür, wie wir die weitere
Nutzung dieses Wissen (Haraway) effektiver ermöglichen können und vor
allem darüber, wie wir Forderungen an die Informationstechnik genauer
formulieren können. Im erwähnten Horizon-Bericht werden neben lösbaren
und schweren auch zwei \enquote{wicked challenges} benannt:
\enquote{Managing Knowledge Obsolescence} (sicher auch der Irrtümer) und
\enquote{Embracing the Need for Radical Change}. Und da ist sie wieder,
die notwendige radikale Änderung! Sie erfordert neue Wertigkeiten, neue
Optiken -- und hier kann auch die Geschlechterforschung ihre Expertise
anbieten:

\begin{enumerate}
\def\labelenumi{\arabic{enumi}.}
\item
  Die Hauptaufgabe der Bibliothek ist die inhaltliche Erschließung.
  (Laut Lehrbuchdefinition für die Bibliothek wohl der synoptische
  Aspekt, denn sammeln, ordnen und zur Verfügung stellen, kann auch ein
  Warenhaus, ob real oder online.)
\item
  Dazu werden die inhaltlichen Erschließungsinstrumente - oder
  Erschließungsaggregate der sich ständig transformierenden
  Wissensproduktion angepasst und damit andro- und eurozentrisches
  Denken überwunden sowie inter- und transdisziplinäres Denken und
  Handeln ermöglicht.
\item
  Wissen ist situiert. Die angebliche wissenschaftliche Objektivität ist
  eine Täuschung. Deshalb sammelt eine Bibliothek verschiedene Wissen
  (\enquote{situiertes Wissen} nach Donna Haraway 1995, 1996) zu einem
  Gebiet, denn: \enquote{Objektivität ist buchstäblich verkörpert}.
  (Barad 2015, S.\,45)
\item
  Bei der transdisziplinären Arbeit sind ständig Übersetzungsleistungen
  nötig. Die Geschlechterforschung hat das zum Beispiel an den
  \enquote{Travelling Concepts} der feministischen Theorie untersucht.
  (Binder u.a. 2011)
\item
  Die zu entwickelnde Software, der Horizon-Bericht nennt
  \enquote{Semantic Web and Linked Data} als mittelfristige
  Entwicklungen in den kommenden zwei bis drei Jahren (S.\,2), muss
  diesen Prozessen angepasst sein und das netzwerkartige Arbeiten in der
  Wissensproduktion mit den Recherchen ermöglichen können.
\end{enumerate}

Dafür schlägt Corinna Bath das \enquote{Diffractive Design} vor als
\enquote{sehr gut geeignet für die Modellierung von Wissen für das
Semantic Web und die LOD Cloud.} Nach ihrer Meinung sind dazu
\enquote{herkömmliche Modellierungsmethoden interferent mit kritischen
Ansätzen der Technikgestaltung, Geschlechterforschung und feministischer
Epistem-(onto-)logie durch einander hindurch zu lesen und auf einen
(Wissens-)Bereich anzuwenden.} (Bath 2013, S.\,110) Diese Texte empfehle
ich allen, die sich für Softwaregestaltung und Techniksoziologie
interessieren, weil sie auch eine mögliche Lösung für die notwendig zu
schaffenden Algorithmen automatischen Schließens sein können, was eine
\enquote{radikal interdisziplinäre Zusammenarbeit von
Geschlechterforscher\_innen mit Technikgestalter\_innen erforderlich
macht}. Es tut sich was auf diesem Gebiet, auch von Seiten der
Geschlechterforschung in der Bibliothekswissenschaft, zum Beispiel im
\enquote{Journal of Academic Librarianship} (Henry 2015) oder im
\enquote{code\{4\}lib Journal} vom April des Jahres 2015, wo die
Autor\_innen die Zukunft von Library Discovery-Systemen mit den sechs
Merkmalen feministischer Software-Interaktion auf Basis der Human
Computer Interaction genauer untersuchen: \enquote{plurality,
self-disclosure, participation, ecology, advocacy and embodyment}.
(Sadler/Bourg 2015, S.\,2)

Wichtiger scheint auch zu werden, spezielle Zugänge zum benötigten
Wissen zu entwickeln, also die Dialektik zwischen Standardisierung und
Differenzierung auszureizen. Das Bild vom Ozean des Wissens, aus dem ich
nur ein Glas Wasser haben möchte, verdeutlicht das vielleicht. In diesem
Sinne hat zum Beispiel die Pädagogische Hochschule in Freiburg eine
virtuelle Gender-Systematik für den Gesamtkatalog angefertigt. Sie
ermöglicht eine standortfreie systematische Suche für dieses
inter-/transdisziplinäre Gebiet.\footnote{\url{https://www.ph-freiburg.de/hochschule/zentrale-einrichtungen/bibliothek/suchen/fachgebiete/gender1.html}.}
Oder die Datenbank Gendermedizin \enquote{GenderMed DB}\footnote{\url{http://gendermeddb.charite.de/}.}:
Hier kann die Suche im Vorfeld auf eine bestimmte Fachrichtung oder
Kategorie eingeschränkt werden.

Und abschließend ein kurzer Blick auf ein Projekt, das die Frauen-,
Lesben- und Genderbibliotheken und -archive des i.d.a.-Dachverbandes im
November 2015, auf ihrer 50. Fachtagung in Luxemburg, veröffentlichten.
Erstmals werden die Bestände von bisher dreißig verschiedenen
Einrichtungen in einem META-Katalog zugänglich gemacht.\footnote{\url{http://www.ida-dachverband.de/home/}.}
Und zwar egal, ob Akte, Nachlass oder Buch, egal ob Poster oder Artikel
aus Sammelband oder Zeitschrift. Und (fast) alles sachgerecht
verschlagwortet! Was uns fehlt zum Glück, ist ein Thesaurus, aber dazu
gibt es Ideen und internationale Anknüpfungsmöglichkeiten. (Schenk 2015)
Vor allem sollen dabei die Relationen zwischen Begriffen im Vordergrund
stehen, nicht die Dichotomie von Ober- und Unterbegriffen. Wie die
Recherchezugänge und Facetten des neuen Katalogs META ankommen, werden
wir testen. Auf jeden Fall nutzt das Konstruktionsteam mit der Software
VuFind, ein Instrument, mit dem vielfältigere Funktionen für die Nutzung
der Resultate selbstständiger entwickelt werden können als bei teuren
Firmen (Henry 2015), \enquote{die die Entwicklung bedarfsgerechter
Services seit Jahrzehnten lähmen}. (Mittelbach 2015, S.\,64)

Wir sind gespannt und wissen gleichzeitig: Wir werden nicht ALLES haben,
also das Ideal. Aber: Spiegelt sich in einem Wassertropfen nicht die
ganze Welt? So gesehen, hat eine Bibliothek doch ALLES! Ständige
Ablieferung von Wissensproduktionen durch Wissenschaft und Praxis,
ständiger Bedarf durch Wissenschaft und Praxis.\,Es kommt darauf an, den
Zugang zum situierten Wissen mit neuem Blick zu organisieren. Wir müssen
uns bewusst sein, dass wir die agentischen Schnitte in die vernetzten,
verschränkten, zusammenhängenden Wissen selbst tun, weil anders kein
Agieren möglich ist. Im Erkenntnisprozess von Hegel kommend, bezeichnete
Marx das Ausblenden dieser Tatsache das Geheimnis der spekulativen
Konstruktion, Bourdieu nennt es die scholastische Sicht. Und Barad
verweist auf die Verschränkung, die Diffraktion und Intra-Aktion.
Stärken wir mit dieser Perspektive die inhaltliche Erschließung und
nähern uns asymptotisch einer transdisziplinär arbeitenden Bibliothek.

\section*{Literatur}

Aleksander, Karin (2014): Die Frau im Bibliothekskatalog. In: LIBREAS.
Library Ideas, 25 (2014).
\url{http://libreas.eu/ausgabe25/02alexander/}.

Aleksander, Karin (2005): Wie werden interdisziplinäre
Gender-Studiengänge mit Literatur versorgt? In: Hauke, Petra (Hrsg.):
Bibliothekswissenschaft - quo vadis? : eine Disziplin zwischen
Traditionen und Visionen ; Programme, Modelle, Forschungsaufgaben.
München : Saur, 2005, S.\,265-284.

Aleksander, Karin (2004): Wie werden Gender Studies-Studierende mit der
notwendigen Literatur versorgt? In: Zentrum für transdisziplinäre
Geschlechterstudien an der HU Berlin (Hrsg.): Geschlechterstudien im
deutschsprachigen Raum : Studiengänge, Erfahrungen, Herausforderungen.
Berlin : trafo Verlag, S.\,136-138.

Ball, Rafael (2009): Wissenschaftskommunikation im Wandel - Bibliotheken
sind mitten drin. In: Hohoff, Ulrich; Knudsen, Per, (Hrsg.): Wissen
bewegen - Bibliotheken in der Informationsgesellschaft. (Zeitschrift für
Bibliothekswesen und Bibliographie, Sonderband ; 96). Frankfurt/M, S.
39-54. \url{http://epub.uni-regensburg.de/2049/1/ubr02636_ocr.pdf}.

Barad, Karen (2015): Verschränkungen. Berlin : Merve (Internationaler
Merve-Diskurs ; 409).

Barad, Karen (2013): Diffraktionen : Differenzen, Kontingenzen und
Verschränkungen von Gewicht. In: Geschlechter Interferenzen :
Wissensformen - Subjektivierungsweisen - Materialisierungen. Münster :
LIT Verlag, S.\,27-67.

Barad, Karen (2012): Agentieller Realismus : über die Bedeutung
materiell-diskursiver Praktiken. Berlin : Suhrkamp.

Bath, Corinna; Meißner, Hanna; Trinkaus, Stephan; Völker, Susanne
(2013): Geschlechter Interferenzen : Wissensformen -
Subjektivierungsweisen - Materialisierungen. Berlin; Münster : LIT
Verlag (Geschlechter Interferenzen ; 1).

Bath, Corinna (2013): Semantic Web und Linked Open Data : von der
Analyse technischer Entwicklungen zum \enquote{Diffractive Design}. In:
Bath, Corinna; Meißner, Hanna; Trinkaus, Stephan; Völker, Susanne
(2013): Geschlechter Interferenzen : Wissensformen -
Subjektivierungsweisen - Materialisierungen. Berlin; Münster : LIT
Verlag, S.\,69-115.

Binder, Beate; Kerner, Ina; Kilian, Eveline; Jähnert, Gabriele; Nickel,
Hildegard Maria (Hrsg.) (2011): Travelling Gender Studies :
grenzüberschreitende Wissens- und Institutionentransfers.\,Münster :
Westfälisches Dampfboot (Forum Frauen- und Geschlechterforschung ; 33).

Capurro, Rafael (2001): Skeptisches Wissensmanagement.
\url{http://www.capurro.de/wm-afta.html}.

Dressel, Gert; Berger, Wilhelm; Heimerl, Katharina; Winiwarter, Verena
(Hrsg.) (2014): Interdisziplinär und transdisziplinär forschen :
Praktiken und Methoden. Bielefeld : Transcript.

Ewert, Gisela; Umstätter, Walther (1997): Lehrbuch der
Bibliotheksverwaltung. Stuttgart : Hiersemann.

Haraway, Donna (1996 {[}1988{]}): Situiertes Wissen : die
Wissenschaftsfrage im Feminismus und das Privileg einer partialen
Perspektive. In: Scheich, Elvira (Hrsg.): Vermittelte Weiblichkeit :
feministische Wissenschafts- und Gesellschaftstheorie. Hamburg :
Hamburger Edition, S.\,217-248.

Haraway, Donna (1995): Die Neuerfindung der Natur : Primaten, Cyborgs
und Frauen. Frankfurt/M. ; New York : Campus-Verlag.

Hark, Sabine (2003): Material Conditions : begrenzte Möglichkeiten
inter- und transdisziplinärer Frauen- und Geschlechterforschung. In:
Zeitschrift für Frauen- und Geschlechterforschung. Bielefeld
21(2003)2+3, S.76-89.

Heintel, Peter (2009): Wege aus der Randständigkeit - ein Brückenschlag.
In: Hanschitz, Rudolf-Christian; Schmidt, Esther; Schwarz, Guido
(Hrsg.): Transdisziplinarität in Forschung und Praxis : Chancen und
Risiken partizipativer Prozesse. Wiesbaden : VS Verlag für
Sozialwissenschaften, S.\,23-197 (Schriften zur Gruppen- und
Organisationsdynamik ; 5).

Henry, Ray Laura (2015): Moving from Theory to Practice : Incorporating
Feminist Approaches into Search and Discovery Tool Development. In: The
Journal of Academic Librarianship, 41(2015)4, S.\,514-516.

Hobohm, Hans-Christoph (2007): Bibliothek(swissenschaft) 2.0 : neue
Auflage oder Wende in Forschung und Lehre? In: LIBREAS.\,Library Ideas,
10/11. \url{http://libreas.eu/ausgabe10/003hob.htm}.

Junger, Ulrike (2015): Quo vadis Inhaltserschließung \ldots{}? :
Herausforderungen und Perspektiven. In: o-bib : das offene
Bibliotheksjournal (VDB). Heft 1, S.\,20.
\url{https://www.o-bib.de/article/view/2015H1S15-26}.

Keller, Alice (2009): Die perfekte Bibliothek. In: B.I.T.online, Heft 2.
\url{http://www.b-i-t-online.de/archiv/2009-02-idx.html}.

Lube, Manfred (Red.) (2000): Die Bibliothek in den Augen der
Universität : eine Befragung der Lehrenden Dezember 1999-Jänner 2000. Klagenfurt : Universitätsbibliothek.
\url{http://ub.uni-klu.ac.at/cms/fileadmin/ub/dokumente/publikationen/fragebogenbericht.pdf}.

Minnich, Elizabeth (1982): Liberal Arts and Civic Arts : Education for
the Free Man?. In: Liberal Education 68(1982)4, S.\,311-322.

Mittelbach, Jens (2015): Modernes Datenmanagement : Linked Open Data und
die offene Bibliothek. In: o-bib : das offene Bibliotheksjournal (VDB).
Heft 2, S.\,64. \url{https://www.o-bib.de/article/view/2015H2S61-73}.

Mittelstraß, Jürgen (2012): Transdisziplinarität oder: von der schwachen
zur starken Interdisziplinarität. In: Gegenworte, Heft 28, S.\,10-13.

Mückenberger, Ulrich; Timpf, Siegfried (2010): Transdisziplinarität als
doppelte Grenzüberschreitung : realexperimentelle Raum-Zeitgestaltung in
urbanen Quartieren. In: Läpple, Dieter; Mückenberger, Ulrich;
Oßenbrügge, Jürgen (Hrsg.): Zeiten und Räume der Stadt : Theorie und
Praxis.\,Opladen : Budrich, S.\,205-228.

Redaktion-LIBREAS (2006): Editorial LIBREAS 4 (Winter 2006) :
Philosophische Fragen in Bibliothek und Bibliothekswissenschaft. In:
LIBREAS.\,Library Ideas, 4 (2006), S.\,1
\url{http://libreas.eu/ausgabe4/000edi.htm}.

Ruppelt, Georg (2005): Warum? : \ldots{} anstelle eines Vorwortes.\,In:
Hauke, Petra (Hrsg.): Bibliothekswissenschaft - quo vadis? : eine
Disziplin zwischen Traditionen und Visionen ; Programme, Modelle,
Forschungsaufgaben. München : Saur, S.\,7-8.

Sadler, Bess; Bourg, Chris (2015): Feminism and the Future of Library
Discovery. In: code\{4\}lib Journal, Issue 28.
\url{http://journal.code4lib.org/articles/10425}.

Jasmin Schenk (2015): Konzept Gender Thesaurus : zur Bedeutung einer
gemeinsamen Dokumentationssprache für Forschung und
Informationseinrichtungen. Köln : Fachhochschule.
\url{http://malisprojekte.web.fh-koeln.de/wordpress/jasmin-schenk-konzept-gender-thesaurus-zur-bedeutung-einer-gemeinsamen-dokumentationssprache-fuer-forschung-und-informationseinrichtungen/}.

Uhlmann, Sandro (2013): Automatische Beschlagwortung von
deutschsprachigen Netzpublikationen mit dem Vokabular der Gemeinsamen
Normdatei (GND). In: Dialog mit Bibliotheken. Leipzig 25(2013)2, S.
26-36.

Weinheimer, James (2014): A Conversation Between a Patron and the
Library Catalog. \url{https://www.youtube.com/watch?v=C7M69J7IwyQ}.

Zimmer, Dieter E. (2000): Die Zukunft der Bibliothek. Hamburg, S.\,258
(Er zitiert Hans H. Wellisch: Indexing from A to Z. New York, 1991).

Alle Internetquellen wurden am 28.08.2015 letztmalig geprüft. 
